%%%%%%%%%%%%%%%%%%%%%%%%%%%%%%%%%%%%%%%%%%%%%%%%%%%%%%%%%%%%%%%%%%%%%%%%%%%%%%%%
% Author : [Name] [Surname], Tomas Polasek (template)
% Description : Seventh exercise in the Introduction to Game Development course.
%   It deals with the creation of a Game Design Document, presenting a short 
%   pitch for a potential game project.
%%%%%%%%%%%%%%%%%%%%%%%%%%%%%%%%%%%%%%%%%%%%%%%%%%%%%%%%%%%%%%%%%%%%%%%%%%%%%%%%

\documentclass[a4paper,10pt,slovak]{article}

\usepackage[left=2.50cm,right=2.50cm,top=1.50cm,bottom=2.50cm]{geometry}
\usepackage[utf8]{inputenc}

% Hyper-Text References
\usepackage{hyperref}
\hypersetup{colorlinks=true, urlcolor=blue}

% Drawing Images and Graphs
\usepackage{tikz}
\usepackage{pgfplots}

% Page Utilities
\usepackage{graphicx}

% Image Sub-Captions
\usepackage{subcaption}

\newcommand{\ph}[1]{\textit{[#1]}}

\title{%
Game Pitch Document%
}
\author{%
Martin Talajka (xtalaj00)%
}
\date{}

\begin{document}

\maketitle
\thispagestyle{empty}

{%
\large

\begin{itemize}

\item[] \textbf{Názov:} Space Cleaner [WIP name]

\item[] \textbf{Žáner:} Survival, Adventure (Action-Adventure), Sandbox?, Simulation

\item[] \textbf{Štýl:} 2D, PixelArt

\item[] \textbf{Platforma:} Windows, Nintendo Switch

\item[] \textbf{Market:} Všetky vekové kategórie, Dlhodobý hráč, Vesmírne dobrodružstvo, Simulátor

\item[] \textbf{Elevator Pitch:} Vyčisti vesmír a vystavaj svoju vlastnú kolóniu z ničoho !

\end{itemize}

}

\section*{\centering The Pitch}

\subsection*{Introduction}
Základom hry by bolo čistenie vesmíru od odpadu. Podarilo sa nám zničiť zem, ktorá je neobývateľná a museli sme sa presťahovať do vesmíru, ktorý je plný starých satelitov, rakiet ale aj pôvodných kozmických telies, ktoré sa dajú zozbierať a využiť z nich potrebné materiály na prežitie. 

\subsection*{Background}
Veľmi ma inšpirovala mapa na ktorej je vidno znečistenie okolo našej planéty (http://stuffin.space/). Taktiež sa mi páči myšlienka ten vesmír vyčistiť a keďže obľubujem hry typu Rimworld, Oxygen Not Included a podobne, myšlienka vybudovať kolóniu na vesmírnom odpade a 2D - pixel art, bola teda jasná voľba. Miernu inšpiráciu som nabral aj z hry Space Engineers, ktorá hre pridala mierne prvky simulácie a technický charakter.

\subsection*{Setting}
Hra by bola založená na príbehu hlavnej postavy, ktorá bola vyslaná do vesmíru spolu s ďalšími členmi posádky, aby založili kolóniu a pripravili zázemie pre zvyšok ľudí. Takýchto kolónií bolo vyslaných niekoľko po celej zemi. Bohužial na zemi dochádzajú suroviny a aj čas na žitie, preto je potrebné aby si všetky suroviny zabezpečili vo vesmíre – čistením odpadu a extrakciou vzácnych kovov, ťažením surovín, obchodovaním s inými kolóniami ..

Bolo by časom obmedzené vybudovať plne samostatnú kolóniu pre záchranu ľudí na zemi. Hra by mala viac koncov? (hra sa teoreticky nedá dohrať – dovŕšime ku koncu ale môžeme ďalej pokračovať v hraní. Rozširovať kolóniu, budovať .. Podstate niečo ako Minecraft). Išlo by o to že môžu všetci na zemi umrieť, ak sa to nestihne v čas a tým pádom nezachránime ľudí, len svoju kolóniu. Alebo sa nám podarí vybudovať stanicu a zachránime ľudí a môžeme pokračovať s nimi. Tak či tak by bola možnosť potom planétu vyčistiť, znovu zalesniť a obývať. Čo sa nám ale nemusí podariť kvôli nedostatku surovín, kolonistov a rôznych iných faktorov.

Podstate by to fungovalo ako v hre Rimworld, bolo by možné budovať kolóniu, obchodovať, rozvíjať vzťahy medzi ľuďmi, pestovať, zabíjať, automatizovať a hlavne čistiť a recyklovať !

\newpage

\subsection*{Features}
Hra má veľmi zaujímavý koncept a veľmi peknú pixel grafiku. Nájdu si v nej záľubu technicky zamerané typy, ľudia ekologicky zameraní, blázni do vesmíru ale aj ľudia ktorí si užívajú príbeh, ktorý je veľmi členitý. Poprípade aj ľudia čo si chcú len oddýchnuť a na pár minút – hodín si zahrať hru.
\newline\newlineS
Space Cleaner ponúka veľmi rôznorodé mechaniky:
\begin{itemize}
    \item Dlhý a rozvinutý príbeh, ktorý sa sám vyvíja podľa toho ako hráč hrá (vie si ho podstate sám postaviť)
    \item Spraviť si množstvo nepriateľov alebo priateľov
    \item Možnosť byť kreatívny a stavať neobmedzene veľké kolónie a stanice
    \item Preskúmať zákutia vesmíru a anomálie, ktoré sa tam vyskytujú
    \item Automatizovať proces, vysielať drony na zbieranie, recyklovanie ..
    \item Vytvárať veľké procesné linky, farmy, ekosystém
    \item Do budúcna budeme podporovať aj multiplayer !
\end{itemize}

\subsection*{Genre}
Survival žáner tejto hre patrí oprávnene, keďže ide viacmenej o záchranu planéty a ľudstva s obmedzeným množstvom surovín. 

V hre sa budeme stretávať s veľa prekážkami, blúdením vesmírom, skúmaním, ťažbou ale aj bránenou sa útokom, kšeftovaniu a podobne .. preto Adventure (Action-Adventure). 

Simulation som sem zaradil kvôli možnosti vytvárania automatizovaných fabrík, liniek a podobne ale taktiež aj o to že sa jedná o mierny simulátor vo vesmíre. 

Sandbox je len taký bočný žáner, vďaka tomu ako je hra situovaná a dá sa hrať do nekonečna, je tu možnosť sandboxu, keď si niekto chce len stavať a zamerať sa viac na túto časť hrania ako Survival element.

\subsection*{Platform}
Prvý plán je hru vydať na platformu Windows. Keďže najväčšia komunita hráčov vlastní zrovna počítač a ja sám mám bližšie k počítačom, tak táto platforma je preferovaná. Neskôr by som hru rád rozšíril aj na Linux, MacOS a ostatné systémy. 

Čo sa konzol týka, hru by som rád rovno vydal aj na Nintendo Switch. Keďže som spomínal že hru bude možné hrať čo i len pár minút, táto platforma je pre mňa tiež veľmi dobrá voľba. Veľa ľudí používa switch napríklad pri cestovaní alebo len na príležitostné hranie. 
Nebránil by som sa vydaniu hry aj na ostatné konzoly ale momentálne to neplánujem. 


\end{document}
