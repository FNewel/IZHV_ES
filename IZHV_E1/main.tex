%%%%%%%%%%%%%%%%%%%%%%%%%%%%%%%%%%%%%%%%%%%%%%%%%%%%%%%%%%%%%%%%%%%%%%%%%%%%%%%%
% Author : [Name] [Surname], Tomas Polasek (template)
% Description : First exercise in the Introduction to Game Development course.
%   It deals with an analysis of a selected title from the point of its genre, 
%   style, and mechanics.
%%%%%%%%%%%%%%%%%%%%%%%%%%%%%%%%%%%%%%%%%%%%%%%%%%%%%%%%%%%%%%%%%%%%%%%%%%%%%%%%

\documentclass[a4paper,10pt,english]{article}

\usepackage[left=2.50cm,right=2.50cm,top=1.50cm,bottom=2.50cm]{geometry}
\usepackage[utf8]{inputenc}
\usepackage[T1]{fontenc}
\usepackage{hyperref}
\hypersetup{colorlinks=true, urlcolor=blue}

\newcommand{\ph}[1]{\textit{[#1]}}

\title{%
Analýza hernej mechaniky%
}
\author{%
Martin Talajka (xtalaj00)%
}
\date{}

\begin{document}

\maketitle
\thispagestyle{empty}

{%
\large

\begin{itemize}

\item[] \textbf{Názov:} Hellblade: Senua's Sacrifice

\item[] \textbf{Dátum vydania:} 2017

\item[] \textbf{Autor:} Ninja Theory

\item[] \textbf{Primárny žáner:} action-adventure

\item[] \textbf{Sekundárny žáner:} hack and slash, puzzle solving, psychological horror, serious game

\item[] \textbf{Štýl:} realistic

\end{itemize}

}

\section*{\centering Analýza}

Túto hru som si na analýzu vybral z niekoľkých dôvodov. Jeden z dôvodov bol, že som si hranie veľmi užil a náramne sa mi páčilo aj jej spracovanie a odkaz ktorý hra mala zanechať. Celá hra je zameraná na Severskú mytológiu, ktorú veľmi obľubujem. To ale nie je dôvod prečo som si hru vybral, ten dôvod je že hra má jeden veľmi dôležitý aspekt, ktorý ju oddeľuje od ostatných hier a zaraďuje ju tak medzi seriózne hry. Celý čas hráme za hlavnú postavu, ktorá trpí schizofréniou. 

Celý gameplay sa odohráva okolo hlavnej postavy Senui, ktorá sa snaží zachrániť dušu svojho mŕtveho muža od bohine Helly. Senua počas cestovania naráža na veľa problémov a počas toho vysvetľuje jej príbeh.  Jedným zo žánrov hry je hack and slash, ktorý je naplno využívaný pri súbojoch s prízrakmi. Zo začiatku Senua svoj zlý psychický stav nezvláda moc dobre a to že počuje hlasy a vidí veci, ktoré nikto iný nevidí a nepočuje, pripisuje k zlým silám ako ju to odmalička učili. To sa ale v priebehu hrania mení a hlavná hrdinka sa naučí tieto hlasy a vidiny používať v svoj prospech. Počas celého hrania (odporúčam hru hrať na slúchadlách) počujeme hlasy, ktoré Senue zvyčajne „napomáhajú“. Veľmi dobre je to spracované napríklad pri súboji, kde tieto hlasy napovedajú odkiaľ sa blíži útok, poprípade kde sú prízraky. Ďalším žánrom hry je puzzle solving. Tento žáner je veľmi pekne zakomponovaný do gameplayu hry, kde tiež využívame Senuine psychické problémy, aby sme videli svet z iného „uhlu“ a otvorili si inak nedostupné cesty. Medzi vedľajšie žánre by som zaradil aj psychological horror, ktorý vďaka pochmúrnej atmosfére, útočiacim prízrakom a napätým momentom počas hrania, právom patrí do zoznamu. A aby toho nebolo málo, počas celého hrania, narážame na malé obelisky, pri ktorých nám rozprávač vysvetľuje Severskú mytológiu.


\subsection*{Zhrnutie}

Celú hru by som zhrnul ako veľmi zábavnú a poučnú. Hra vyrozpráva veľmi napätý a pekný príbeh, pri ktorom je prerozprávaná celá Severská mytológia ako vedľajší príbeh a ešte priblíži schizofrenickú psychózu. Hlavný žáner hry sa veľmi dobre doplňuje s vedľajšími žánrami a celá mechanika hry funguje výborne. Po grafickej ale aj hudobnej stránke hry by sme ju mohli zaradiť veľmi vysoko. A vďaka vývojárom spolupracujúcim s neurovedcami, odborníkmi na duševné zdravie a ľuďmi ktorí s týmto ochorením žijú, je spracovanie tejto psychózy veľmi verné a dobre spracované/zakomponované v hre.

\end{document}
